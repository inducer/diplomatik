#parse ("tema-hd-statistik-kopf.tex")
\begin{document}
  \begin{center}
    \textbf{Studienabschl\"usse in}\\\\
    \textbf{$h.escapeTeX($drs.description())}

    im Zeitraum 
    $h.formatDate($form_data.From)
    --
    $h.formatDate($form_data.To)
  \end{center}


  \begin{longtable}[l]{|c|c|c|c|c|c|c|c|c|c|c|}
  \hline
  \textit{Name}
  & \textit{Nr}
  & \textit{GN}
  & \textit{DA}
  & \textit{Betreuer}
  & \textit{MATH}
  & \textit{Info.}
  & \multicolumn{2}{|c|}{\textit{Tech.Fach}}
  & \textit{Pr\"ufz.}
  & \textit{Sem.}\\\\
  \hline
  \endhead
  #set ($number = 0)
  #foreach ($sd in $stud_deg)##
    #set ($number = $h.add($number, 1))##
    $sd.Student.LastName ##
    & $number
    & $h.formatNumber("%.1f", $sd.Gesamt)
    & $h.formatNumber("%.1f", $sd.Diplomarbeit.CountedResult)
    & $sd.Diplomarbeit.Examiner
    & $h.formatNumber("%.1f", $sd.Mathematik)
    & $h.formatNumber("%.1f", $sd.TechFach)
    & $h.formatNumber("%.1f", $sd.Informatik)
    & $sd.TechFachName
    & $sd.Pruefungszeitraum
    & $sd.Semesters#if ($sd.SpecialSemesters)+$sd.SpecialSemesters#end
    \\\\
    \hline
  #end##
  \end{longtable}

  \textbf{Erl\"auterungen der Abk\"urzungen:}
  \vskip0.3cm

  \begin{tabular}{lcl}
    \textit{GN} & = & Gesamtnote \\\\
    \textit{DA} & = & Diplomarbeit \\\\
    \textit{MATH} & = & Durchschnittsnote der drei mathematischen F\"acher \\\\
    \textit{RM} & = & Reine Mathematik \\\\
    \textit{AM} & = & Angewandte Mathematik \\\\
    \textit{Info.} & = & Angewandte Informatik \\\\
    \textit{Tech. Fach.} & = & Technisches Fach \\\\
    \textit{Pr\"ufz.} & = & Zeit in Monaten zwischen
      Abgabe der Diplomarbeit und letzter Pr\"ufung \\\\
    \textit{Sem.} & = & Studiendauer in Semestern \\\\
  \end{tabular}

  #parse ("tema-hd-statistik-generisch.tex")
\end{document}
