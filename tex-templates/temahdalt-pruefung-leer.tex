\input header

\def\VORNAME{$student.FirstName}
\def\NAME{$student.LastName}
\def\MATRIKEL{$student.ID}
\def\TECHFACH{$degree.MinorSubject}
\def\PRUEFORDDAT{$h.formatDate($drs.ruleSetDate())}
\def\PRUEFORDPARA{\S 10 (2)}


\begin{document}
\begin{tabular}{ccrl}
{\bf UNIVERSIT\"AT KARLSRUHE (TH)}& \hspace{3.4cm}  & Karlsruhe,
den \makebox[2.0cm]{\dotfill\hspace*{0.25cm}} \\\\
{\bf Fakult\"at f\"ur Mathematik } & & &   \\\\
\end{tabular}

\vspace*{1.3cm}

{\large Herrn/Frau Professor/in \dotfill
\vspace*{0.4cm}

\begin{center}
{\Large\underline{Zulassung zur Diplom--Hauptpr\"ufung TECHNOMATHEMATIK}}\\
\end{center}
\vspace*{0.5cm}

{\large Name, Vorname: \makebox[7.2cm]{\dotfill\qh{\tt\NAME,
 \VORNAME}\qh\dotfill}\hspace{0.25cm}
 Matrikelnummer: \hfill{\tt\MATRIKEL}}
\vspace*{0.3cm}

\def\Box{\circ}%

\renewcommand{\arraystretch}{1.4}
\begin{center}
\begin{tabular}{|c|p{14.6cm}|}
\hline
\hspace*{0.3cm} 
& {\large Reine Mathematik} \\\\
\hline
& {\large Angewandte Mathematik }\\\\
\hline 
& {\large Technisches Fach: {\tt\TECHFACH}}\\\\
  %$\qquad \Box$ Bauingenieurwesen, $\qquad \Box$ Elektrotechnik,}\\
  %& {\large $\qquad\Box$ Maschinenbau, $\qquad \Box$ Mechanik,
  %$\qquad \Box$ Physik }\\
\hline
& {\large Angewandte Informatik}\\\\
\hline
& {\large Zusatzfach} \\\\
\hline
\end{tabular}
\end{center}
\renewcommand{\arraystretch}{1}
\normalsize

Sie werden gebeten, im angekreuzten Fach die Pr\"ufung (SWS)\\\\

\dotfill \\\\

bis zum \makebox[4cm][c]{\dotfill} abzunehmen.

\vspace*{0.35cm}

\hspace*{7cm} \dotfill \\\\
\hspace*{8.5cm} {\small F\"ur den Pr\"ufungsausschuss{} f\"ur
 Mathematik}\\\\
\hspace*{10.0cm} {\small Prof.~Dr.~W. D\"orfler} \\\\

\fbox{\fbox{\parbox{15.5cm}{

\vspace*{0.2cm}

   {\quad \bf \underline{Bewertung der Pr\"ufung}:}

\vspace*{0.5cm}

\qquad \dotfill \qquad \\\\
\qquad \hspace*{\fill} (Note) \hfill \hfill (in Worten)
     \hfill \hfill (Datum der Pr\"ufung) \hspace*{\fill} \\\\

\qquad \qquad
  \dotfill \dotfill \qquad \qquad \\\\
  \hspace*{\fill} (Unterschrift der Pr\"ufer bzw. Pr\"ufer und
Beisitzer) \hspace*{\fill}
 }}}

\vspace*{0.3cm}
[\PRUEFORDPARA\ der Pr\"ufungsordnung vom \PRUEFORDDAT: \\\\
Die Leistungen in den einzelnen F\"achern sind mit den folgenden Noten
zu bewerten: \\\\
\hspace*{3cm} {\bf 1} = sehr gut; \hspace*{2cm} {\bf 2} = gut; \hspace*{2cm}
{\bf 3} = befriedigend; \\\\
\hspace*{3cm} {\bf 4} = ausreichend; \hspace*{1.4cm} {\bf 5} =
nicht ausreichend. \\\\
Im Zeugnis d\"urfen nur diese Noten verwendet werden. Die Notenziffern
im Protokoll \\\\
k\"onnen zur Differenzierung um 0.3 erh\"oht oder erniedrigt werden.
Die Noten \\\\
0.7, \ 4.3 \ und \ 5.3 \ sind ausgeschlossen.] \\\\
\rule{15.8cm}{0.03in} \\\\
%\rule{15.8cm}{0.03in} \\
Bitte senden Sie dieses Blatt sofort nach der Pr\"ufung an den
Verantwortlichen f\"ur Technomathematik im Pr\"ufungsausschu\ss{} f\"ur
Mathematik:

\vspace*{-0.3cm}
\begin{tabbing}
{\bf Adresse:} \ \= Prof. Dr. W. D\"orfler, Inst.~Angewandte Mathematik II, \\\\
\> Universit\"at Karlsruhe, Englerstr.\ 2 (Geb.~20.30), 76128
Karlsruhe.
\end{tabbing}
\end{document}
