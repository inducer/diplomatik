\renewcommand{\baselinestretch}{1.1}

\begin{center}
{\large AUSFERTIGUNG F\"UR DIE PR\"UFUNGSABTEILUNG UND DEN PR\"UFUNGSAUSSCHUSS\\
DIPLOMHAUPTPR\"UFUNG TECHNOMATHEMATIK}

{\bf (Pr\"uf.-Ordn.\ vom \PRUEFORDDAT)}
\end{center}
\vspace*{0.4cm}

NAME, VORNAME: \makebox[7.2cm]{\dotfill\qh{\tt\NAME,
 \VORNAME}\qh\dotfill}\hspace{0.25cm}
 Matrikelnummer: \hfill{\tt\MATRIKEL}

geb.~am
\makebox[5.0cm]{\dotfill\qh{\tt\GEBDAT}\qh\dotfill}\hspace{0.25cm}
 in\hspace{0.25cm}\dotfill\qh{\tt\GEBORT}\qh\dotfill

Beginn des Studiums:
\makebox[4.5cm]{\dotfill\qh{\tt\STUDBEG}\qh\dotfill}\hspace{0.25cm}
 Diplomvorpr\"ufung: \dotfill\qh{\tt\VORDIPL}\qh\dotfill

Zahl der Fachsemester bei Abschlu\ss{} der Hauptpr\"ufung:
 \makebox[6.0cm]{\dotfill\FACHSEMESTER\URLAUBSSEMESTERTEXT\dotfill}
\vspace*{0.2cm}

% Notenwoerter berechnen
\renewcommand{\arraystretch}{1.5}
\begin{center}
\begin{tabular}{p{4.6cm}|c|c|p{2.0cm}|p{4.3cm}}
{\bf PR\"UFUNGSF\"ACHER} & {\bf PR\"UF.--} & {\bf IN WORTEN} &
{\bf\qh DATUM} & {\bf\qh PR\"UFER} \\[-0.3cm]
                & {\bf NOTEN} &  &  & \\
\hline
{\bf Reine Mathematik} & {\tt\RMnoteZ} & {\tt\RMnoteW} & {\tt\RMdatum} & {\tt\RMpruefer} \\[8pt]
\hline
{\bf Angew.~Mathematik} & {\tt\AMnoteZ} & {\tt\AMnoteW} & {\tt\AMdatum} & {\tt\AMpruefer} \\[8pt]
\hline
{\bf Techn.~Fach {\tt\TECHFACH}} & {\tt\TFnoteZ} & {\tt\TFnoteW} & {\tt\TFdatum} & {\tt\TFpruefer} \\[8pt]
\hline
{\bf Angew.~Informatik} & {\tt\AInoteZ} & {\tt\AInoteW} & {\tt\AIdatum} & {\tt\AIpruefer} \\[8pt]
\hline
{\bf Zusatzfach} {\tt\ZUSFACH} & {\tt\ZFnoteZ} & {\tt\ZFnoteW} & {\tt\ZFdatum} & {\tt\ZFpruefer} \\[8pt]
\hline
\end{tabular}
\end{center}
\renewcommand{\arraystretch}{1}
\normalsize

{\bf \underline{Diplomarbeit}}

\begin{center}
   {\tt\DAtitel}\\
   (Abgegeben am {\tt\DAabgabe})
\end{center}

\begin{tabular}{p{5.0cm}p{10.0cm}}
   {\bf Note:}\q\DAnote & {\bf Referent:}\q\DAref
%   {\bf Note:}\makebox[2cm]{\DAnote}\makebox[4cm]{}{\bf Referent:}\q\DAref
\end{tabular}

\rule{17cm}{0.03in} \\

% Die Abschlussrechnung
\FPset{\Gnote}{\RMnoteZ}%
\FPadd{\Gnote}{\Gnote}{\AMnoteZ}
\FPadd{\Gnote}{\Gnote}{\TFnoteZ}
\FPadd{\Gnote}{\Gnote}{\AInoteZ}
\FPmul{\DAnote}{2}{\DAnote}\FPround{\DAnote}{\DAnote}{1}
\FPadd{\Gnote}{\Gnote}{\DAnote}
\FPtrunc{\Gnote}{\Gnote}{1}\FPtrunc{\Enote}{\Enote}{1}

\begin{tabular}{lcp{2.2cm}p{3cm}p{6cm}}
\multicolumn{4}{l}{\bf Berechnung des Gesamturteils} &
{\bf Hauptpr\"ufung abgeschlossen} \\[0.2cm]
Reine Math.\ & (1x) & \makebox[2cm]{\dotfill\RMnoteZ\dotfill}
             & \hspace*{3cm} & am \dotfill\Abschlussdatum\dotfill \\
Angew. Math.\ & (1x) & \makebox[2cm]{\dotfill\AMnoteZ\dotfill}
              &\hspace*{3cm} & mit dem Gesamturteil \\
Techn.~Nebenfach & (1x) & \makebox[2cm]{\dotfill\TFnoteZ\dotfill}
& \hspace*{3cm} & \\
Angew.~Informatik & (1x) & \makebox[2cm]{\dotfill\AInoteZ\dotfill} &
&\hrulefill\qh\texttt{\large\EnoteW}\qh\hrulefill \\
Diplomarbeit & (2x) &  \makebox[2cm]{\dotfill\DAnote\dotfill}
\begin{minipage}[t]{2.2cm}
\makebox[2cm]{\hrulefill}%$= = = = = = =$
\end{minipage}
& & F\"ur den Pr\"ufungs\-ausschuss f\"ur die Diplompr\"ufung
in Mathematik \\
\\
& & \makebox[0.8cm]{\hrulefill}\qh\Gnote\qh\makebox[0.8cm]{\hrulefill}
& : 6 =\makebox[0.4cm]{\hrulefill}\qh\textbf{\large\Enote}
\qh\makebox[0.4cm]{\hrulefill} & \makebox[6cm]{\dotfill}\\
& & & & \hspace{1.5cm}{\small Prof.~Dr.~W. D\"orfler}
\end{tabular}

\vspace*{0.3cm}

\rule{17cm}{0.02in} \\
{\bf\underline{Bemerkungen}:}\\
\Bemerkung
\newpage
