\documentclass[landscape, 10pt]{article}
\batchmode
\usepackage[latin1]{inputenc}
\usepackage{longtable}

%%%%%%%%%%%%%%%%%%%%%%%%%%%%%%%%%%%%%%%%%%%%%%%%%%%%%%%%%%%%%%%%%%%%%%%%
\def\q{\quad}%
\def\qh{$\;$}%
%%%%%%%%%%%%%%%%%%%%%%%%%%%%%%%%%%%%%%%%%%%%%%%%%%%%%%%%%%%%%%%%%%%%%%%%
\textheight250mm%
\textwidth160mm%
%%%%%%%%%%%%%%%%%%%%%%%%%%%%%%%%%%%%%%%%%%%%%%%%%%%%%%%%%%%%%%%%%%%%%%%%
\setlength{\parindent}{0cm}%
\setlength{\topmargin}{-1.0cm}%
\setlength{\headheight}{0cm}%
\setlength{\headsep}{0cm}%
\setlength{\oddsidemargin}{0.0cm}%
\setlength{\evensidemargin}{0.0cm}%
\pagestyle{empty}%
\begin{document}
  \begin{center}
    \textbf{Studienabschl\"usse in TECHNOMATHEMATIK (Diplom)}

    im Zeitraum 
    $h.formatDate($form_data.From)
    --
    $h.formatDate($form_data.To)
  \end{center}


  \begin{longtable}[l]{|c|c|c|c|c|c|c|c|c|c|c|c|c|}
  \hline
  \textit{Name}
  & \textit{Nr}
  & \textit{GN}
  & \textit{DA}
  & \textit{Betreuer}
  & \textit{PN}
  & \textit{RM}
  & \textit{AM}
  & \textit{Info.}
  & \multicolumn{2}{|c|}{\textit{Tech.Fach}}
  & \textit{Pr\"ufz.}
  & \textit{Sem.}\\\\
  \hline
  \endhead
  #set ($number = 0)
  #foreach ($sd in $stud_deg)##
    #set ($number = $h.add($number, 1))##
    $sd.Student.LastName ##
    & $number
    & $h.formatNumber("%.1f", $sd.Gesamt)
    & $h.formatNumber("%.1f", $sd.Diplomarbeit.CountedResult)
    & $sd.Diplomarbeit.Examiner
    & $h.formatNumber("%.1f", $sd.Pruefungsnoten)
    & $h.formatNumber("%.1f", $sd.Rein)
    & $h.formatNumber("%.1f", $sd.Angewandt)
    & $h.formatNumber("%.1f", $sd.NF2)
    & $h.formatNumber("%.1f", $sd.NF1)
    & $sd.NF1Name
    & $sd.Pruefungszeitraum
    & $sd.Semesters#if ($sd.SpecialSemesters)+$sd.SpecialSemesters#end
    \\\\
    \hline
  #end##
  \end{longtable}

  \textbf{Erl\"auterungen der Abk\"urzungen:}
  \vskip0.3cm

  \begin{tabular}{lcl}
    \textit{GN} & = & Gesamtnote \\\\
    \textit{DA} & = & Diplomarbeit \\\\
    \textit{PN} & = & Durchschnittsnote der vier 
      Pr\"ufungen \\\\
    \textit{RM} & = & Reine Mathematik \\\\
    \textit{AM} & = & Angewandte Mathematik \\\\
    \textit{Info.} & = & Angewandte Informatik \\\\
    \textit{Tech. Fach.} & = & Technisches Fach \\\\
    \textit{Pr\"ufz.} & = & Zeit in Monaten zwischen
      Abgabe der Diplomarbeit und letzter Pr\"ufung \\\\
    \textit{Sem.} & = & Studiendauer in Semestern \\\\
  \end{tabular}

  \newpage
  \begin{center}
    \textbf{Studiengang TECHNOMATHEMATIK (Diplom)}

    Statistik f\"ur den Zeitraum 
    $h.formatDate($form_data.From)
    --
    $h.formatDate($form_data.To)
  \end{center}

  \begin{tabular}{llll}
    \textbf{Anzahl der Abschl\"usse:} 
    & insgesamt & $h.len($stud_deg) \\\\
    & m\"annlich & $m_w.__getitem__("m") \\\\
    & weiblich & $m_w.__getitem__("w") \\\\
    \\\\
    \textbf{Gesamtnote:}
    & mit Auszeichnung & $gesamt_hist.__getitem__("mit Auszeichnung") \\\\
    & sehr gut & $gesamt_hist.__getitem__("sehr gut") \\\\
    & gut & $gesamt_hist.__getitem__("gut") \\\\
    & befriedigend & $gesamt_hist.__getitem__("befriedigend") \\\\
    \\\\
    \textbf{Note der Diplomarbeit:}
    #foreach ($da_grade in $h.sort($da_hist.keys())) ##
      & $h.formatNumber("%.1f", $da_grade) 
      & $da_hist.__getitem__($da_grade) \\\\
    #end ##
    \\\\
    \textbf{Notendurchschnitt der vier Pr\"ufungen:}
    #foreach ($pn in $h.sort($pn_hist.keys())) ##
      & $h.formatNumber("%.1f", $pn) 
      & $pn_hist.__getitem__($pn) \\\\
    #end ##
    \\\\
    \textbf{Studiendauer:}
      & Median & $sem_median & Semester \\\\
    \\\\
    \textbf{Pr\"ufungszeitraum:}
      & Median & $pz_median & Monate \\\\
    \\\\
    \textbf{Nebenfach:}
    #foreach ($nf in $h.sort($nf_hist.keys())) ##
      & $nf
      & $nf_hist.__getitem__($nf) \\\\
    #end ##
    \\\\
    \textbf{Anzahl der gewerteten studienbegl. Pr\"ufungen:}
    #foreach ($sb in $h.sort($sb_hist.keys())) ##
      & $sb Pr\"ufungen
      & $sb_hist.__getitem__($sb)
      & Studierende(r) \\\\
    #end ##
    \\\\
    \textbf{Pr\"ufungen aus dem Ausland:}
      & brachte ein
      & $ausland_count
      & Studierende(r)
  \end{tabular}
\end{document}
